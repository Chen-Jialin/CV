% !TEX program = xelatex
\documentclass[letterpaper,11pt]{article}
\usepackage[scheme=plain,linespread=1,punct=CCT]{ctex}% Chinese support, single line space, narrow-version SBC case punctuations
\usepackage[hmargin=.75in,vmargin=.4in]{geometry}
\pagestyle{empty}% Suppress page numbers
\usepackage[parfill]{parskip}% Remove paragraph indentation
% Bold Songti
\setCJKfamilyfont{zhsong}[AutoFakeBold={2.17}]{SimSun}
% \renewcommand{\songti}{\CJKfamily{zhsong}}
%
% English font
\usepackage{fontspec}
\setmainfont{Times New Roman}
%
% Narrow vertical space of itemize
\usepackage{enumitem}
\setlist{itemsep=0pt,partopsep=0pt,parsep=0pt,topsep=0ex}
%
\usepackage{hyperref}% 超链接
\begin{document}
\begin{center}
    {\LARGE\bfseries{}陈稼霖}\\
    \vspace{1ex}
    安徽省合肥市金寨路 96 号, 230026\\
    chenjialin@mail.ustc.edu.cn | chenjl@shanghaitech.edu.cn | (+86) 158-5868-9289\\
    \href{https://github.com/Chen-Jialin}{github.com/Chen-Jialin}\\
\end{center}

{\Large\bfseries{}教育背景}\\
\rule[1.5ex]{\textwidth}{1pt}
{\songti\large\bfseries{}上海科技大学}{\large\quad{}物理学\quad{}理学学士}\hfill{2017.09 -- 2021.07}\\
\vspace{-4ex}
\begin{itemize}
    \item 绩点:{\bfseries{}3.88/4.0};专业排名:1/38;学院排名:1/91
    \item 优秀本科生奖学金(二等奖 2 次,三等奖 1 次),优秀毕业生
    \item 相关课程:激光原理技术、飞秒激光与超快光谱技术、非线性光学、通信原理、导波光学
\end{itemize}
{\songti\large\bfseries{}中国科学技术大学}{\large\quad{}光学工程\quad{}工程硕士}\hfill{2021.09 至今}\\
\vspace{-1ex}% \vspace{1ex}

{\Large\bfseries{}科研经历}\\
\rule[1.5ex]{\columnwidth}{1pt}\\
{\songti\large\bfseries{}上海科技大学 John A. McGuire 教授课题组}\hfill{2020.06 -- 2021.07}\\
\vspace{-4ex}
\begin{itemize}
    \item 涉猎二维电子光谱(2D ES)相关知识
    \item 协助搭建二维电子光谱系统,编写用于控制线扫描 CMOS 传感器的 LabVIEW 程序;优化用于数据处理的 LabVIEW 程序模块,节约部分模块运行耗时超 50\%
\end{itemize}
{\songti\large\bfseries{}上海科技大学~颜世超~教授课题组}\hfill{2018.07 -- 2020.05}\\
\vspace{-4ex}
\begin{itemize}
    \item 涉猎扫描隧道显微镜(STM)相关知识
    \item 利用 SolidWorks 设计两套扫描隧道显微镜针尖刻蚀装置的支架,用于安全、高效地刻蚀针尖
    \item 绘制整个实验室及其内部设施的立体图,以便实验室内设备的规划和安装
\end{itemize}
\vspace{1ex}

{\Large\bfseries{}竞赛获奖}\\
\rule[1.5ex]{\columnwidth}{1pt}\\
\vspace{-4ex}
\begin{itemize}
    \item {\songti\large\bfseries{}美国大学生数学建模竞赛(MCM/ICM)},{\large{}Honorable Mention}\hfill{2020.02}
    \item {\songti\large\bfseries{}全国大学生数学建模竞赛},{\large{}上海市二等奖}\hfill{2019.12}
    \item {\songti\large\bfseries{}中国大学生物理学术竞赛(CUPT)}:与另一位搭档共同负责“Moiré Thread Countor”“Hurricane Ball”“Soy Sauce Optics”“Invent Yourself”共计 4 个课题,累计上场比赛 5 次,获奖 3 次:
    \begin{itemize}
        \item[$\circ$] {\large{}全国赛二等奖}\hfill{2019.08}
        \item[$\circ$] {\large{}上海市赛二等奖}\hfill{2019.07}
        \item[$\circ$] {\large{}华东地区赛二等奖}\hfill{2019.05}
    \end{itemize}
\end{itemize}
\vspace{1ex}

{\Large\bfseries{}校内工作经历}\\
\rule[1.5ex]{\columnwidth}{1pt}\\
{\songti\large\bfseries{}上海科技大学物质科学与技术学院},{\large{}原子物理中的量子力学实验~~课程助教}\hfill{2021.03 -- 2021.06}\\
\vspace{-4ex}
\begin{itemize}
    \item 随堂指导学生实验
\end{itemize}
{\songti\large\bfseries{}上海科技大学物质科学与技术学院},{\large{}物理学术竞赛创新实验~~课程助教}\hfill{2020.07 -- 2020.08}\\
\vspace{-4ex}
\begin{itemize}
    \item 课程辅讲,实验答疑
\end{itemize}
% {\songti\large\bfseries{}上海科技大学教学发展中心},{\large{}行政助管}\hfill{2020.05 -- 2021.01}\\
% \vspace{-4ex}
% \begin{itemize}
%     \item 处理本科教学管理相关的图表绘制、问卷撰写、文件翻译等文书工作
%     \item 负责历次教学研讨会中的来宾接待和登记工作
% \end{itemize}
{\songti\large\bfseries{}上海科技大学物质科学与技术学院},{\large{}物理原理 I 实验~~课程助教}\hfill{2019.09 -- 2020.01}\\
\vspace{-4ex}
\begin{itemize}
    \item 教授物理专业一年级本科生MATLAB、Adobe Illustrator、Excel、Tracker等软件及高速摄像机的使用
    \item 指导学生进行“液体表面张力测量”实验,批改实验报告
    \item 协助授课教师指导学生完成“液滴的显微放大效果”“振动的光学放大”两个探究性课题
\end{itemize}
\vspace{1ex}

{\Large\bfseries{}技能及证书}\\
\rule[1.5ex]{\columnwidth}{1pt}\\
\vspace{-4ex}
\begin{itemize}
    \item \textbf{\songti\bfseries{}英语水平}:四级:629;六级:578;TOEFL:100
    \item {\songti\bfseries{}软件和编程语言}:Python(全国计算机等级考试二级合格证书)、MATLAB、LabVIEW、Fortran、\LaTeX、Adobe Illustrator、SolidWorks
    \item GRE:328+3.5;GRE Subject (Physics):970
    % \item 上海红十字会基础急救培训证书
\end{itemize}
\end{document}